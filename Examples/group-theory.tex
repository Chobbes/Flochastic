\documentclass{article}

\usepackage{amsmath}
\usepackage{amssymb}
\usepackage{amsthm}

\setlength{\parindent}{0pt}
\setlength{\parskip}{2ex}

\newtheorem{lemma}{Lemma}
\newtheorem{theorem}{Theorem}
\newtheorem{prop}{Proposition}
\newtheorem{mdef}{Definition}

\newcommand{\normal}{\trianglelefteq}
\DeclareMathOperator{\im}{im}
\DeclareMathOperator{\aut}{Aut}

\author{
  Beck, Calvin\\
}

\title{Group Theory}

\begin{document}
\maketitle

\part*{2014-09-04:}

\begin{mdef}[Group]
A group $G$ is a set together with a binary operation $\cdot : G \times G \to G$. These must abide by the group axioms:

\begin{enumerate}
\item \textbf{Associativity:} $(a \cdot b) \cdot c = a \cdot (b \cdot c), \qquad \forall a,b \in G$
\item \textbf{Identity:} $\exists e \in G$ such that $e \cdot a = a \cdot e = a, \qquad \forall a \in G$
\item \textbf{Inverses:} $\forall a \in G, \exists b \in G$ such that $a \cdot b = b \cdot a = e$. We say that $b = a^{-1}$
\end{enumerate}
\end{mdef}

\begin{mdef}[Subgroup]
Let $G$ be a group. A subset $H \subseteq G$ is a subgroup if $H$ is also a group. It sufficies to show that:

\begin{enumerate}
\item $e \in H$
\item $H$ is closed under the group operation
\item If $h \in H$, then $h^{-1} \in H$
\end{enumerate}

If $H$ is a subgroup of $G$ we write $H \leq G$.
\end{mdef}

\begin{lemma}
  Assume $G$ is a finite group, and that $H \leq G$ such that $2 \cdot |H| = |G|$, then $H \normal G$.

  \begin{proof}
    $H \normal G \implies aHa^{-1} = H, \quad \forall a \in G$

    Clearly $a \in H \implies aHa^{-1} = H$, thus we shall focus on the case where $a \not \in H$.

    If $a \not \in H$ then $aH \cap H = \emptyset$, since if $ah_1 = h_2$ for some $h_1, h_2 \in H$ then $a = h_2 h_1^{-1} \in H$, which contradicts our assumption that $a \not \in H$. This also means $|aH| = |H|$, and thus $G = aH \sqcup H$, since $|G| = 2 \cdot |H| = |aH \sqcup H|$.

    Clearly $Ha \cap H = \emptyset$ by the same argument, which implies that $Ha = aH$. Thus $H \normal G$.
  \end{proof}
\end{lemma}

\begin{mdef}[Homomorphism]
  A homomorphism is a map between groups $\varphi : G \to H$ such that

  \begin{enumerate}
  \item $\varphi(e_G) = e_H$
  \item $\varphi(g_1 g_2) = \varphi(g_1) \varphi(g_2)$
  \end{enumerate}
\end{mdef}

\begin{mdef}[Kernel]
  For a homomomorphism $\varphi : G \to H$, we define the kernel

  \[\ker \varphi = \{g \in G : \varphi(g) = e_H\}\]
\end{mdef}

\begin{mdef}[Image]
  The image of a homomorphism $\varphi : G \to H$ is given by

  \[\im \varphi = \{h \in H : \exists g \in G, \varphi(g) = h\} = \varphi(G)\]
\end{mdef}

\begin{lemma}
  The kernel of $\varphi : G \to H$ is normal in $G$.

  \begin{proof}
    Let $N = \ker \varphi$, and $n \in N$. For any $g \in G$

    \begin{align*}
      \varphi(gng^{-1}) &= \varphi(g) \varphi(n) \varphi(g^{-1})\\
      &= \varphi(g) \varphi(g^{-1})\\
      &= \varphi(g g^{-1})\\
      &= e_H
    \end{align*}

    Thus $N \normal G$.
  \end{proof}
\end{lemma}

\begin{lemma}
  $I = \im \varphi$ is a group.

  \begin{proof}
    Must show that $I$ contains an identity, inverses, and is closed under the group operation.

    \begin{enumerate}
    \item Homomorphisms preserve identity.
    \item $i \in I$, then $i = \varphi(g)$ for some $g \in
      G$. $\varphi(g^{-1}) \varphi(g) = e_I \implies \varphi(g^{-1}) =
      i^{-1} \in I$.
    \item $i_1, i_2 \in I$, and $i_1 = \varphi(g_1)$, $i_2 =
      \varphi(g_2)$ for some $g_1, g_2 \in G$.

    \begin{align*}
      i_1 i_2 &= \varphi(g_1) \varphi(g_2)\\
      &= \varphi(g_1 g_2) \in I
    \end{align*}
    \end{enumerate}

   Thus $\im \varphi$ is a group.
  \end{proof}
\end{lemma}

\part*{2014-09-09}

\begin{lemma}
  If $H, K \leq G$, then $H \cap K \leq K$

  \begin{proof}
    Let $I = H \cap K$.

    \begin{enumerate}
    \item $e \in I$ since both $H$ and $K$ must have the identity in order to be groups.
    \item $\forall i \in I$, $i^{-1} \in I$ since

      \begin{align*}
        i \in H &\implies i^{-1} \in H\\
        i \in K &\implies i^{-1} \in K
      \end{align*}

      Thus, $i^{-1} \in I$.

    \item $\forall i_1, i_2 \in I$,

      \[i_1, i_2 \in H \implies i_1 i_2 \in H\]

      and similarly

      \[i_1, i_2 \in K \implies i_1 i_2 \in K\]

      Thus, $H \cap K$ is closed under the group operation.
    \end{enumerate}
  \end{proof}
\end{lemma}

\begin{mdef}
  Let $H, K \leq G$

  \[H \cdot K = \{h \cdot k : h \in H, k \in K\}\]
\end{mdef}

\begin{lemma}
  Let $H, K \leq G$.

  \[HK \leq G \iff HK = KH\]

  \begin{proof}
    Suppose first that $HK \leq G$. That is

    \[HK = \{hk : h \in H, k \in K\}\]

    is a subgroup of $G$. We want to show that $\forall hk \in HK$, $\exists k'h' \in KH$ such that $hk = k'h'$. Since $HK$ is a group $(hk)^{-1} = k^{-1} h^{-1} \in HK$, thus let $(hk)^{-1} = {h'}^{-1} {k'}^{-1}$. It is then clear that $hk = k'h' \in KH$, and thus $HK = KH$.

    Now suppose that $HK = KH$.

    \begin{enumerate}
    \item $HK$ must contain the identity element since both $H$ and $K$ are groups.
    \item $hk \in HK, k^{-1} h^{-1} \in KH$, $HK = KH \implies k^{-1} h^{-1} \in HK$.
    \item $h_1 k_1, h_2 k_2 \in HK$, then

      \[h_1 k_1 h_2 k_2 = h_1 h_2' k_1' k_2\]

      for some $h_2' \in H, k_1' \in K$, where $h_2' k_1' = k_1 h_2$.
    \end{enumerate}
  \end{proof}
\end{lemma}

\begin{mdef}[Normalizer]
  Let $K \leq G$. The normalizer of $K$ in $G$

  \[N_G(K) = \{g \in G : gKg^{-1} = K\}\]
\end{mdef}

\begin{lemma}
  Given $K \leq G$, $N_G(K)$ is a group.

  \begin{proof}
    Recall $N_G(K) = \{g \in G : gKg^{-1} = K\}$.

    \begin{enumerate}
    \item Clearly $N_G(K)$ contains the identity.
    \item $g \in N_G(K)$. $gKg^{-1} = K \implies g^{-1} K g = K$, thus $g^{-1} \in N_G(K)$.
    \item $a, b \in N_G(K)$. $abK(ab)^{-1} = abKb^{-1}a^{-1} = aKa^{-1} = K$.
    \end{enumerate}
  \end{proof}
\end{lemma}

\begin{lemma}
  If $H \subseteq N_G(K)$, then $HK \leq G$.

  \begin{proof}
    $\forall h \in H$, $hK = Kh \implies HK = KH$.
  \end{proof}
\end{lemma}

\begin{mdef}[External Direct Product]
  Let $H, K$ be two groups. The direct product of $H$ and $K$ is

  \[H \times K = \{(h, k) : h \in H, k \in K\}\]

  Where multiplication is defined componentwise

  \[(h_1, k_1) \cdot (h_2, k_2) = (h_1 h_2, k_1 k_2)\]

  We also note that

  \begin{align*}
    H &\cong \{(h, 1) : h \in H\}\\
    K &\cong \{(1, k) : k \in K\}\\
  \end{align*}
\end{mdef}

\begin{mdef}[Internal Direct Product]
  Given a group $G$ and $H, K \leq G$ when is $G = H \times K$?

  \begin{enumerate}
    \item $H \cap K = 1$
    \item $HK = G$
    \item $H, K \normal G$
  \end{enumerate}

  \begin{proof}
    We want an isomorphism $\varphi : H \times K \to G$, $(h, k) \mapsto (h \cdot k)$.

    First we need this to be a homomorphism.

    \begin{enumerate}
    \item $\varphi(1, 1) = 1 \cdot 1 = 1$
    \item $(h_1, k_1), (h_2, k_2) \in H \times K$

      \[\varphi((h_1, k_1)(h_2, k_2)) = \varphi(h_1 h_2, k_1 k_2) = h_1 h_2 k_1 k_2\]

      and similarly

      \[\varphi((h_1, k_1)(h_2, k_2)) = \varphi(h_1, k_1) \varphi(h_2, k_2) = h_1 k_1 h_2 k_2\]

      This means $h_2 k_1 = k_1 h_2$, and since these are arbitrary these groups must commute with each other (and since $HK = G$ these are normal in $G$).
    \end{enumerate}

    If this is surjective  $\im \varphi = G$. This can only happen if $HK = G$.

    Finally, we need $\ker \varphi = \{1\}$ in order to show that the map is injective. If this were not the case then $\varphi(h, k) = h k = 1$ for some $h \in H$ and $k \in K$. However, this would mean that $h = k^{-1}$ and it is therefore the case that $h, k \in H \cap K$.

    Thus, if $\varphi$ is injective then $H \cap K = 1$.
  \end{proof}
\end{mdef}

\begin{mdef}[Automorphism Groups]
  An automorphism is a homomorphism $\varphi : G \to G$. The automorphism group of a group $G$ is defined as:

  \[\aut(G) = \{\varphi : G \to G : \varphi \text{ is a homomorphism}\}\]
\end{mdef}

\begin{mdef}[Conjugation Automorphism]
  \[C_g(h) = g h g^{-1}\]
\end{mdef}


\end{document}
